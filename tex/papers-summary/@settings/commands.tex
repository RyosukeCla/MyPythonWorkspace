\usepackage{xparse}
\usepackage{pgfkeys}

% NewBibItemArgs
\pgfkeys{
  /NewBibItemArgs/.is family, /NewBibItemArgs,
  % Here are the options that a user can pass
  default/.style = {
    author = Auther,
    title = Title,
    booktitle = BookTitle,
    month = Jun,
    year = 2018,
  },
  % Here is the dictionary for positions.
  author/.estore in = \bibAuthor,
  title/.estore in = \bibTitle,
  booktitle/.estore in = \bibBookTitle,
  month/.estore in = \bibMonth,
  year/.estore in = \bibYear,
}

% NewBibItem{cite key}{NewBibItemArgs}
\NewDocumentCommand {\NewBibItem} { m m } {%
  \pgfkeys{/NewBibItemArgs, default, #2}%
  \bibitem{#1} \bibAuthor. \bibTitle. \textit{\bibBookTitle}. \bibMonth, \bibYear.
}

% InProceedings{cite key}{NewBibItemArgs}
\NewDocumentCommand {\InProceedings} { m m } {%
  \pgfkeys{/NewBibItemArgs, default, #2}%
  \bibitem{#1} \bibAuthor. \bibTitle. In \textit{Proceedings of \bibBookTitle}. \bibMonth, \bibYear.
}


% PutImageInCenter{image path in @imgs}{label}[Option{A Picture}][Option{width=8cm}][Option{h}]
\NewDocumentCommand {\PutImageInCenter} { m m O{A Picture} O{width=8cm} O{h} } {%
  \begin{figure}[#5]
    \includegraphics[#4]{#1}
    \centering
    \caption{#3}
    \label{#2}
  \end{figure}
}
