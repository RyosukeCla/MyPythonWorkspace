\documentclass[12pt,fleqn,dvipdfmx]{jarticle}
  \usepackage{docmute}
  \documentclass[12pt,fleqn,dvipdfmx]{jarticle}

  \documentclass[12pt,fleqn,dvipdfmx]{jarticle}

  \documentclass[12pt,fleqn,dvipdfmx]{jarticle}

  \input{../@settings/index}
  \input{../@settings/commands}

  \begin{document}
  \section{STARGAN}
  Unified Generative Adversarial Networks for Multi-Domain Image-to-Image Translation
  \subsection{Abstract}
  \subsection{Intro}
  Hoges \cite{StarGAN2018}

  \newpage
  \begin{thebibliography}{99}
    \InProceedings{StarGAN2018}{%
      author={Choi, Yunjey and Choi, Minje and Kim, Munyoung and Ha, Jung-Woo and Kim, Sunghun and Choo, Jaegul},
      title={StarGAN: Unified Generative Adversarial Networks for Multi-Domain Image-to-Image Translation},
      booktitle={The IEEE Conference on Computer Vision and Pattern Recognition (CVPR)},
      month={June},
      year={2018}
    }
  \end{thebibliography}

  \end{document}

  \usepackage{xparse}
\usepackage{pgfkeys}

% NewBibItemArgs
\pgfkeys{
  /NewBibItemArgs/.is family, /NewBibItemArgs,
  % Here are the options that a user can pass
  default/.style = {
    author = Auther,
    title = Title,
    booktitle = BookTitle,
    month = Jun,
    year = 2018,
  },
  % Here is the dictionary for positions.
  author/.estore in = \bibAuthor,
  title/.estore in = \bibTitle,
  booktitle/.estore in = \bibBookTitle,
  month/.estore in = \bibMonth,
  year/.estore in = \bibYear,
}

% NewBibItem{cite key}{NewBibItemArgs}
\NewDocumentCommand {\NewBibItem} { m m } {%
  \pgfkeys{/NewBibItemArgs, default, #2}%
  \bibitem{#1} \bibAuthor. \bibTitle. \textit{\bibBookTitle}. \bibMonth, \bibYear.
}

% InProceedings{cite key}{NewBibItemArgs}
\NewDocumentCommand {\InProceedings} { m m } {%
  \pgfkeys{/NewBibItemArgs, default, #2}%
  \bibitem{#1} \bibAuthor. \bibTitle. In \textit{Proceedings of \bibBookTitle}. \bibMonth, \bibYear.
}


% PutImageInCenter{image path in @imgs}{label}[Option{A Picture}][Option{width=8cm}][Option{h}]
\NewDocumentCommand {\PutImageInCenter} { m m O{A Picture} O{width=8cm} O{h} } {%
  \begin{figure}[#5]
    \includegraphics[#4]{#1}
    \centering
    \caption{#3}
    \label{#2}
  \end{figure}
}


  \begin{document}
  \section{STARGAN}
  Unified Generative Adversarial Networks for Multi-Domain Image-to-Image Translation
  \subsection{Abstract}
  \subsection{Intro}
  Hoges \cite{StarGAN2018}

  \newpage
  \begin{thebibliography}{99}
    \InProceedings{StarGAN2018}{%
      author={Choi, Yunjey and Choi, Minje and Kim, Munyoung and Ha, Jung-Woo and Kim, Sunghun and Choo, Jaegul},
      title={StarGAN: Unified Generative Adversarial Networks for Multi-Domain Image-to-Image Translation},
      booktitle={The IEEE Conference on Computer Vision and Pattern Recognition (CVPR)},
      month={June},
      year={2018}
    }
  \end{thebibliography}

  \end{document}

  \usepackage{xparse}
\usepackage{pgfkeys}

% NewBibItemArgs
\pgfkeys{
  /NewBibItemArgs/.is family, /NewBibItemArgs,
  % Here are the options that a user can pass
  default/.style = {
    author = Auther,
    title = Title,
    booktitle = BookTitle,
    month = Jun,
    year = 2018,
  },
  % Here is the dictionary for positions.
  author/.estore in = \bibAuthor,
  title/.estore in = \bibTitle,
  booktitle/.estore in = \bibBookTitle,
  month/.estore in = \bibMonth,
  year/.estore in = \bibYear,
}

% NewBibItem{cite key}{NewBibItemArgs}
\NewDocumentCommand {\NewBibItem} { m m } {%
  \pgfkeys{/NewBibItemArgs, default, #2}%
  \bibitem{#1} \bibAuthor. \bibTitle. \textit{\bibBookTitle}. \bibMonth, \bibYear.
}

% InProceedings{cite key}{NewBibItemArgs}
\NewDocumentCommand {\InProceedings} { m m } {%
  \pgfkeys{/NewBibItemArgs, default, #2}%
  \bibitem{#1} \bibAuthor. \bibTitle. In \textit{Proceedings of \bibBookTitle}. \bibMonth, \bibYear.
}


% PutImageInCenter{image path in @imgs}{label}[Option{A Picture}][Option{width=8cm}][Option{h}]
\NewDocumentCommand {\PutImageInCenter} { m m O{A Picture} O{width=8cm} O{h} } {%
  \begin{figure}[#5]
    \includegraphics[#4]{#1}
    \centering
    \caption{#3}
    \label{#2}
  \end{figure}
}


  \begin{document}
  \section{STARGAN}
  Unified Generative Adversarial Networks for Multi-Domain Image-to-Image Translation
  \subsection{Abstract}
  \subsection{Intro}
  Hoges \cite{StarGAN2018}

  \newpage
  \begin{thebibliography}{99}
    \InProceedings{StarGAN2018}{%
      author={Choi, Yunjey and Choi, Minje and Kim, Munyoung and Ha, Jung-Woo and Kim, Sunghun and Choo, Jaegul},
      title={StarGAN: Unified Generative Adversarial Networks for Multi-Domain Image-to-Image Translation},
      booktitle={The IEEE Conference on Computer Vision and Pattern Recognition (CVPR)},
      month={June},
      year={2018}
    }
  \end{thebibliography}

  \end{document}

  \usepackage{xparse}
\usepackage{pgfkeys}

% NewBibItemArgs
\pgfkeys{
  /NewBibItemArgs/.is family, /NewBibItemArgs,
  % Here are the options that a user can pass
  default/.style = {
    author = Auther,
    title = Title,
    booktitle = BookTitle,
    month = Jun,
    year = 2018,
  },
  % Here is the dictionary for positions.
  author/.estore in = \bibAuthor,
  title/.estore in = \bibTitle,
  booktitle/.estore in = \bibBookTitle,
  month/.estore in = \bibMonth,
  year/.estore in = \bibYear,
}

% NewBibItem{cite key}{NewBibItemArgs}
\NewDocumentCommand {\NewBibItem} { m m } {%
  \pgfkeys{/NewBibItemArgs, default, #2}%
  \bibitem{#1} \bibAuthor. \bibTitle. \textit{\bibBookTitle}. \bibMonth, \bibYear.
}

% InProceedings{cite key}{NewBibItemArgs}
\NewDocumentCommand {\InProceedings} { m m } {%
  \pgfkeys{/NewBibItemArgs, default, #2}%
  \bibitem{#1} \bibAuthor. \bibTitle. In \textit{Proceedings of \bibBookTitle}. \bibMonth, \bibYear.
}


% PutImageInCenter{image path in @imgs}{label}[Option{A Picture}][Option{width=8cm}][Option{h}]
\NewDocumentCommand {\PutImageInCenter} { m m O{A Picture} O{width=8cm} O{h} } {%
  \begin{figure}[#5]
    \includegraphics[#4]{#1}
    \centering
    \caption{#3}
    \label{#2}
  \end{figure}
}


  \begin{document}

  \title{papers summary}
  \author{Ryosuke Suzuki}
  \date{}
  \maketitle
  \tableofcontents
  \newpage

  % papers

  \documentclass[12pt,fleqn,dvipdfmx]{jarticle}

  \documentclass[12pt,fleqn,dvipdfmx]{jarticle}

  \documentclass[12pt,fleqn,dvipdfmx]{jarticle}

  \input{../@settings/index}
  \input{../@settings/commands}

  \begin{document}
  \section{STARGAN}
  Unified Generative Adversarial Networks for Multi-Domain Image-to-Image Translation
  \subsection{Abstract}
  \subsection{Intro}
  Hoges \cite{StarGAN2018}

  \newpage
  \begin{thebibliography}{99}
    \InProceedings{StarGAN2018}{%
      author={Choi, Yunjey and Choi, Minje and Kim, Munyoung and Ha, Jung-Woo and Kim, Sunghun and Choo, Jaegul},
      title={StarGAN: Unified Generative Adversarial Networks for Multi-Domain Image-to-Image Translation},
      booktitle={The IEEE Conference on Computer Vision and Pattern Recognition (CVPR)},
      month={June},
      year={2018}
    }
  \end{thebibliography}

  \end{document}

  \usepackage{xparse}
\usepackage{pgfkeys}

% NewBibItemArgs
\pgfkeys{
  /NewBibItemArgs/.is family, /NewBibItemArgs,
  % Here are the options that a user can pass
  default/.style = {
    author = Auther,
    title = Title,
    booktitle = BookTitle,
    month = Jun,
    year = 2018,
  },
  % Here is the dictionary for positions.
  author/.estore in = \bibAuthor,
  title/.estore in = \bibTitle,
  booktitle/.estore in = \bibBookTitle,
  month/.estore in = \bibMonth,
  year/.estore in = \bibYear,
}

% NewBibItem{cite key}{NewBibItemArgs}
\NewDocumentCommand {\NewBibItem} { m m } {%
  \pgfkeys{/NewBibItemArgs, default, #2}%
  \bibitem{#1} \bibAuthor. \bibTitle. \textit{\bibBookTitle}. \bibMonth, \bibYear.
}

% InProceedings{cite key}{NewBibItemArgs}
\NewDocumentCommand {\InProceedings} { m m } {%
  \pgfkeys{/NewBibItemArgs, default, #2}%
  \bibitem{#1} \bibAuthor. \bibTitle. In \textit{Proceedings of \bibBookTitle}. \bibMonth, \bibYear.
}


% PutImageInCenter{image path in @imgs}{label}[Option{A Picture}][Option{width=8cm}][Option{h}]
\NewDocumentCommand {\PutImageInCenter} { m m O{A Picture} O{width=8cm} O{h} } {%
  \begin{figure}[#5]
    \includegraphics[#4]{#1}
    \centering
    \caption{#3}
    \label{#2}
  \end{figure}
}


  \begin{document}
  \section{STARGAN}
  Unified Generative Adversarial Networks for Multi-Domain Image-to-Image Translation
  \subsection{Abstract}
  \subsection{Intro}
  Hoges \cite{StarGAN2018}

  \newpage
  \begin{thebibliography}{99}
    \InProceedings{StarGAN2018}{%
      author={Choi, Yunjey and Choi, Minje and Kim, Munyoung and Ha, Jung-Woo and Kim, Sunghun and Choo, Jaegul},
      title={StarGAN: Unified Generative Adversarial Networks for Multi-Domain Image-to-Image Translation},
      booktitle={The IEEE Conference on Computer Vision and Pattern Recognition (CVPR)},
      month={June},
      year={2018}
    }
  \end{thebibliography}

  \end{document}

  \usepackage{xparse}
\usepackage{pgfkeys}

% NewBibItemArgs
\pgfkeys{
  /NewBibItemArgs/.is family, /NewBibItemArgs,
  % Here are the options that a user can pass
  default/.style = {
    author = Auther,
    title = Title,
    booktitle = BookTitle,
    month = Jun,
    year = 2018,
  },
  % Here is the dictionary for positions.
  author/.estore in = \bibAuthor,
  title/.estore in = \bibTitle,
  booktitle/.estore in = \bibBookTitle,
  month/.estore in = \bibMonth,
  year/.estore in = \bibYear,
}

% NewBibItem{cite key}{NewBibItemArgs}
\NewDocumentCommand {\NewBibItem} { m m } {%
  \pgfkeys{/NewBibItemArgs, default, #2}%
  \bibitem{#1} \bibAuthor. \bibTitle. \textit{\bibBookTitle}. \bibMonth, \bibYear.
}

% InProceedings{cite key}{NewBibItemArgs}
\NewDocumentCommand {\InProceedings} { m m } {%
  \pgfkeys{/NewBibItemArgs, default, #2}%
  \bibitem{#1} \bibAuthor. \bibTitle. In \textit{Proceedings of \bibBookTitle}. \bibMonth, \bibYear.
}


% PutImageInCenter{image path in @imgs}{label}[Option{A Picture}][Option{width=8cm}][Option{h}]
\NewDocumentCommand {\PutImageInCenter} { m m O{A Picture} O{width=8cm} O{h} } {%
  \begin{figure}[#5]
    \includegraphics[#4]{#1}
    \centering
    \caption{#3}
    \label{#2}
  \end{figure}
}


  \begin{document}
  \section{STARGAN}
  Unified Generative Adversarial Networks for Multi-Domain Image-to-Image Translation
  \subsection{Abstract}
  \subsection{Intro}
  Hoges \cite{StarGAN2018}

  \newpage
  \begin{thebibliography}{99}
    \InProceedings{StarGAN2018}{%
      author={Choi, Yunjey and Choi, Minje and Kim, Munyoung and Ha, Jung-Woo and Kim, Sunghun and Choo, Jaegul},
      title={StarGAN: Unified Generative Adversarial Networks for Multi-Domain Image-to-Image Translation},
      booktitle={The IEEE Conference on Computer Vision and Pattern Recognition (CVPR)},
      month={June},
      year={2018}
    }
  \end{thebibliography}

  \end{document}


  \end{document}

  % 画像挿入用
  % \begin{figure}[htbp]\centering\includegraphics[width=9cm, bb=0 0 300 300]{photo.pdf}\caption{}\label{fig:photo}\end{figure}

  % 複数画像挿入用
  % \begin{figure}[htbp]\begin{tabular}{c}\begin{minipage}{0.5 \hsize}\centering\includegraphics[width=5cm, bb=0 0 640 640]{photo.jpg}\caption{}\label{fig:}\end{minipage}
  % \begin{minipage}{0.5 \hsize}\centering\includegraphics[width=5cm, bb=0 0 640 640]{photo.jpg}\caption{}\label{fig:}\end{minipage}\end{tabular}\end{figure}

  % 表挿入用
  % \begin{table}[htbp]\centering\caption{}\includegraphics[width=9cm, bb=0 0 300 300]{photo.pdf}\label{tab:photo}\end{table}

  % 参考文献用
  % \begin{thebibliography}{99}
  % \bibitem{bib}中村和幸\ (2014),「題名」,応用数理学会(またはURL+月日アクセス).
  % \end{thebibliography}

  % 参考文献用 2
  % \newpage
  % \begin{thebibliography}{99}
  % \bibitem{toyoda}豊田秀樹\ (2015),「基礎からのベイズ統計: ハミルトニアンモンテカルロ法による実践的入門」,朝倉書店.
  % \bibitem{motomura}本村陽一,岩崎弘利\ (2006),「ベイジアンネットワーク技術\ ユーザ・顧客のモデル化と不確実性推論」,東京電機大学出版局.
  % \end{thebibliography}
