\documentclass[12pt,fleqn,dvipdfmx]{jarticle}

  \input{../@settings/common}
  \input{../@settings/paper}
  \usepackage{xparse}
\usepackage{pgfkeys}

% NewBibItemArgs
\pgfkeys{
  /NewBibItemArgs/.is family, /NewBibItemArgs,
  % Here are the options that a user can pass
  default/.style = {
    author = Auther,
    title = Title,
    booktitle = BookTitle,
    month = Jun,
    year = 2018,
  },
  % Here is the dictionary for positions.
  author/.estore in = \bibAuthor,
  title/.estore in = \bibTitle,
  booktitle/.estore in = \bibBookTitle,
  month/.estore in = \bibMonth,
  year/.estore in = \bibYear,
}

% NewBibItem{cite key}{NewBibItemArgs}
\NewDocumentCommand {\NewBibItem} { m m } {%
  \pgfkeys{/NewBibItemArgs, default, #2}%
  \bibitem{#1} \bibAuthor. \bibTitle. \textit{\bibBookTitle}. \bibMonth, \bibYear.
}

% InProceedings{cite key}{NewBibItemArgs}
\NewDocumentCommand {\InProceedings} { m m } {%
  \pgfkeys{/NewBibItemArgs, default, #2}%
  \bibitem{#1} \bibAuthor. \bibTitle. In \textit{Proceedings of \bibBookTitle}. \bibMonth, \bibYear.
}


% PutImageInCenter{image path in @imgs}{label}[Option{A Picture}][Option{width=8cm}][Option{h}]
\NewDocumentCommand {\PutImageInCenter} { m m O{A Picture} O{width=8cm} O{h} } {%
  \begin{figure}[#5]
    \includegraphics[#4]{#1}
    \centering
    \caption{#3}
    \label{#2}
  \end{figure}
}


  \begin{document}
  \section{STARGAN}
    Unified Generative Adversarial Networks for Multi-Domain Image-to-Image Translation \cite{StarGAN2018}
  \subsection{Abstract}
    Image to Multi Domain ImageができるGAN。経験則的に顔の部分と表情の変換には効果的なモデル。
  \subsection{Intro}
    \subsubsection{Dataset}
      \begin{itemize}
        \item CelebA\cite{CelebA} : 10,177人のセレブ、202,599サイズ、40種類の表情のデータセット
        \item RaFD\cite{RaFD} : 67人の8種類の表情のデータセット
      \end{itemize}
    \subsubsection{Compare}
      既存のmulti domainモデルは、$k$個のドメインに対して$k(k-1)$個のgeneratorを学習させる必要がある。が、StarGANは一個だけでいいんだよ
      (\ref{fig:StarGAN-1})。

      \PutImageInCenter{./stargan-fig-1.png}{fig:StarGAN-1}[Compare]

  \newpage
  \begin{thebibliography}{99}
    \InProceedings{StarGAN2018}{%
      author={Choi, Yunjey and Choi, Minje and Kim, Munyoung and Ha, Jung-Woo and Kim, Sunghun and Choo, Jaegul},
      title={StarGAN: Unified Generative Adversarial Networks for Multi-Domain Image-to-Image Translation},
      booktitle={The IEEE Conference on Computer Vision and Pattern Recognition (CVPR)},
      month={June},
      year={2018}
    }

    \InProceedings{CelebA}{%
      author = {Ziwei Liu and Ping Luo and Xiaogang Wang and Xiaoou Tang},
      title = {Deep Learning Face Attributes in the Wild},
      booktitle = {International Conference on Computer Vision (ICCV)},
      month = December,
      year = {2015}
    }

    \bibitem{RaFD}
      Langner, O., Dotsch, R., Bijlstra, G., Wigboldus, D.H.J., Hawk, S.T., \& van Knippenberg, A.
      (2010). Presentation and validation of the Radboud Faces Database.
      \textit{Cognition \& Emotion, 24(8)}, 1377—1388. DOI: 10.1080/02699930903485076
  \end{thebibliography}

  \end{document}
