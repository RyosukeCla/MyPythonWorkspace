\documentclass[12pt,fleqn,dvipdfmx]{jarticle}
  \usepackage[top=20truemm,bottom=20truemm,left=20truemm,right=20truemm]{geometry}
  \usepackage{setspace}
  \usepackage{amssymb}
  \usepackage{amsmath}
  \usepackage{ascmac}
  \usepackage{theorem}
  \usepackage{enumerate}
  % \usepackage{mathrsfs}
  \usepackage[usenames]{color}
  % \usepackage{graphicx}
  \setstretch{1.25}
  \theorembodyfont{\normalfont}
  \newtheorem{theo}{定理}[section]
  \newtheorem{defi}{定義}[section]
  \newtheorem{lemma}{補題}[section]
  \newtheorem{prop}{命題}[section]
  \newtheorem{coro}{系}[section]
  \newtheorem{exam}{例}[section]
  \newtheorem{notice}{注意}[section]
  \newtheorem{exer}{演習}[section]
  \newtheorem{ques}{問題}[section]
  \newtheorem{answ}{解答}[section]
  \makeatletter
  \renewcommand{\theequation}{% 式番号の付け方
  \thesection.\arabic{equation}}
  \@addtoreset{equation}{section}
  \renewcommand{\thefigure}{% 図番号の付け方
  \thesection.\arabic{figure}}
  \@addtoreset{figure}{section}
  \renewcommand{\thetable}{% 表番号の付け方
  \thesection.\arabic{table}}
  \@addtoreset{table}{section}
  \makeatother
  \allowdisplaybreaks[4]
  \begin{document}
  \title{Q2中村研\\〜ベイジアンネットワーク〜}
  \author{中村和幸}
  \date{}
  \maketitle
  \tableofcontents
  \newpage
  \section{0609:基本方針だよ}
  <流れ>
  \begin{itemize}
    \item 統計の基礎理論を進めていく.\cite{toyoda}を参考にして進めていく.第3章の内容がメインかな.第1,2章は確率・統計の復習が大半なので簡単に流したいな.
    \item その後に,ベイジアンネットワークを進めていく.今の所\cite{motomura}を使用するつもりであるが,ベイズ推定の基礎を固めいているうちにもっと良い参考書を見つけておきたい.
  \end{itemize}

  \section{0616:ベイズの定理}
  \subsection{ベイズの定理}
  ベイズの定理
  \begin{equation}
      P(A|B)=\frac{P(B|A)P(A)}{P(B)}
  \end{equation}
   あういぇい

  \newpage
  \begin{thebibliography}{99}
  \bibitem{toyoda}豊田秀樹\ (2015),「基礎からのベイズ統計: ハミルトニアンモンテカルロ法による実践的入門」,朝倉書店.
  \bibitem{motomura}本村陽一,岩崎弘利\ (2006),「ベイジアンネットワーク技術\ ユーザ・顧客のモデル化と不確実性推論」,東京電機大学出版局.
  \end{thebibliography}

  \end{document}

  %画像挿入用
  %\begin{figure}[htbp]\centering\includegraphics[width=9cm, bb=0 0 300 300]{photo.pdf}\caption{}\label{fig:photo}\end{figure}

  %複数画像挿入用
  %\begin{figure}[htbp]\begin{tabular}{c}\begin{minipage}{0.5 \hsize}\centering\includegraphics[width=5cm, bb=0 0 640 640]{photo.jpg}\caption{}\label{fig:}\end{minipage}
  %\begin{minipage}{0.5 \hsize}\centering\includegraphics[width=5cm, bb=0 0 640 640]{photo.jpg}\caption{}\label{fig:}\end{minipage}\end{tabular}\end{figure}

  %表挿入用
  %\begin{table}[htbp]\centering\caption{}\includegraphics[width=9cm, bb=0 0 300 300]{photo.pdf}\label{tab:photo}\end{table}

  %参考文献用
  %\begin{thebibliography}{99}
  %\bibitem{bib}中村和幸\ (2014),「題名」,応用数理学会(またはURL+月日アクセス).
  %\end{thebibliography}
